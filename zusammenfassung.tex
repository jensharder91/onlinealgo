\documentclass[a4paper]{article}

\usepackage{geometry}
\geometry{margin=3cm,top=2cm,bottom=3cm}

\usepackage[ngerman]{babel}
\usepackage[utf8]{inputenc}
\usepackage{amsmath}
\usepackage{amssymb}
\usepackage{graphicx}

\setlength{\parskip}{\baselineskip}
\setlength{\parindent}{0pt}

\title{Klausurvorbereitung\\Online-Algorithmen}

\author{Jens Harder, Daniel Rauber}

\date{}

\begin{document}
\maketitle

\textbf{Definition 2.8} Ein randomisierter Online-Algorithmus $A$ für ein Minimierungsproblem $\Pi$ erreicht einen kompetitiven Faktor von $r \ge 1$, wenn es eine Konstante $r \in \mathbb{R}$ gibt, sodass
$$E[w_A(\sigma)] \leq r \cdot \textrm{OPT}(\sigma) + \tau$$
für alle Instanzen $\sigma \in I_\Pi$ gilt. Gilt diese Ungleichung sogar für $\tau = 0$, so ist $A$ strikt $r$-kompetitiv.

Der Unterschied zu Definition 1.1 besteht lediglich darin, dass wir hier den Erwartungswert der Kosten nutzen, da diese vom Zufall abhängen.

\textbf{Theorem 2.9} Es sei $A$ ein Online-Algorithmus und $r \geq 1$. Gibt es eine Konstante $b \geq 0$ und eine Potentialfunktion $\Phi$, die die folgenden drei Bedingungen für jede Eingabe $\sigma$ erfüllt, so erreicht Algorithmus $A$ einen kompetitiven Faktor von $r$.
\begin{enumerate}
\item Für jedes $i \geq 1$ gilt $E[a_i] \leq r \cdot \textrm{OPT}_i$.\\
(Die amortisierten Kosten sind höchstens $r$ mal größer als die Kosten von OPT.)
\item Es gilt $E[\Phi_0] \leq b$.\\
(Der Kontostand ist zu Beginn durch eine Konstante nach oben beschränkt.)
\item Für jedes $i \geq 1$ gilt $E[\Phi_i] \geq -b$.\\
(Zu keinem Zeitpunkt wird das Konto stärker als eine Konstante überzogen.)
\end{enumerate}

\textit{Beweisidee:} Da der Startkontostand bzw. das Überziehen des Kontos durch eine Konstante beschränkt sind, kann dies durch das $\tau$ in der Ungleichung des kompetitiven Faktors ausgeglichen werden.

\end{document}
